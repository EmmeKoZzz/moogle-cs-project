\documentclass[a4paper,12pt]{article}
\usepackage[utf8]{inputenc}
\usepackage[T1]{fontenc}
\usepackage[default]{comfortaa}
\usepackage[spanish]{babel}
\usepackage{graphicx}
\usepackage{geometry}

\title{Moogle! Report}
\author{Javier Mustelier Garrido}

\begin{document}

\newgeometry{left=1in,right=1in,top=3cm,bottom=2cm}

% Presentation

\begin{center}
	\thispagestyle{empty}
	\addtocounter{page}{-2} 

	\fontsize{35}{1}
	\bfseries Proyecto Moogle!

	\vspace{1cm}
	
	\Large Javier Mustelier Garrido C122

	\vspace{3cm}

	\includegraphics[width=125pt]{Logo UH.jpeg}

	\vspace{3cm}

	\includegraphics[width=300pt]{matcom.png}

	\vspace{2cm}

	Universidad de la Habana 
	
	\large MATCOM - Facultad de Matemática y Computación
	
\end{center}
\restoregeometry
\normalsize

\begin{center}
	\thispagestyle{empty}
	\tableofcontents
	\clearpage
\end{center}

\section{Introduccion}

El objetivo principal de este proyecto es desarrollar un programa capaz de realizar una busqueda, de forma eficaz y rápida, dentro de un extenso conjunto de documentos de texto plano, de aquellos documentos mas relevantes con respecto a una consulta. Utiliza una interfaz web, desarrollada sobre la tecnologia Razor utilizada en .Net, para brindar facilidad de uso al usuario; mientras q el motor de busqueda esta escrito puramente en el lenguaje C\#. Para lograr desarrollar este motor de busqueda se utilizaron varios metodos en solucion a las problematicas q tiene un proyecto de este tipo, entre ellas el uso del modelo vectorial y la medida TF-IDF, o el uso del algoritmo de Levenshtei; destacando la importancia del algebra lineal en la busqueda de soluciones a problemas capaces de optimizar y ahorrar tiempo en la vida cotidiana.

El programa no solo es realiza una rapida y eficiente busqueda de informacion, también presenta varias funciones q ayudan a elegir y filtrar, y en resumen hacer aún mas preciso y cómodo el proceso de busqueda; para lograr esto se incluyen varios operadores de busqueda en la consulta y sugerencias de mejores consultas con el objetivo de corregir algún posible error en la escritura.

\end{document}